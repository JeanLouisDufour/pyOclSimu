\documentclass[a4paper,10pt]{article} % use larger type; default would be 10pt

\usepackage[a4paper,left=2cm,right=2cm,top=2.5cm,bottom=2.5cm]{geometry}

\bibliographystyle{plainurl}

%\input{../cours_thermo/lib_jld.tex}
%\graphicspath{{./images/}}
\usepackage{hyperref}

%%% The "real" document content comes below...

\title{A certification-oriented OpenCL subset}
\author{Jean-Louis Dufour}
%\date{} % Activate to display a given date or no date (if empty),
         % otherwise the current date is printed 
\date{Safran Electronics \& Defense\\
\today
}

\begin{document}
\maketitle

\begin{abstract}
Manycores are emerging in embedded systems, but only in non-critical usages.
Indeed, the correctness justifications required by a certification process are completely beyond the reach of industrial practice,
and at the frontier of the academic state of the Art.
We focus on a subset of the problem, the correctness of the parallel algorithm.
We emphasize that it has nothing to do with the corresponding problem on multicores,
and propose a justification strategy based on a subset of the OpenCL language.
\\\\
\textbf{Keywords:}\ Manycore, OpenCL, data race, barrier divergence
\end{abstract}



\section{Introduction}

Manycores (i.e. mainly GPUs programmed in CUDA or OpenCL) are emerging in embedded systems, but only in non-critical usages.
Indeed, the correctness justifications required by a certification process are completely beyond the reach of industrial practice,
and at the frontier of the academic state of the Art.
The problem can be decomposed as follows:
\begin{description}
\item[Hardware] is a complex black box, where critical mechanisms like the network-on-chip linking cores to memory banks must be guessed from the patents  \cite{aamodt2018general}; certification-friendly suppliers are a rare exception \cite{boyer2018computing}.
\item[Software / Compiler-and-Scheduler] is also a complex black box, with very few justifications of the compliance w.r.t. the CUDA or OpenCL specification (beyond passing the OpenCL conformance test suite),
\item[Software / Application] (a.k.a. \emph{(compute) kernel}) consists of tens, hundreds or thousands of lightweight threads sharing two levels of memory ('local' and 'global' in OpenCL). This kind of software is the favorite playground of \emph{heisenbugs}, hence test-based methodologies are notoriously ineffective, whether to debug or to verify.
\end{description}
These three aspects are equally important, but the first two are more 'industrial' than technical, because of the competitive nature of the main market of GPUs : video games. We focus here on the third aspect, which is essentially technical, but has also an industrial facet.

Our proposal can be summed up very simply :
\begin{enumerate}
\item we demonstrate that the kernel is \emph{deterministic} : it performs betwen inputs and outputs a transfer \emph{function} and not a transfer \emph{relation}. The only way to do this is to do it \emph{formally}, but it doesn't require the mastery of a formal language: only simple assertions, automatically checkable by an SMT-solver.
\item we then demonstrate that the (parallel) kernel has an equivalent sequential version : it can be done by test.
\item finally, it only remains to validate the sequential version functionally: business as usual (by test).
\end{enumerate}
Heisenbugs are detected by the first step, and after only 'standard' bugs remain.
A formal method is mandatory only for this first step, which will be illustrated on a representative 'real-life' kernel:  the OpenCV histogram equalization function 'equalizeHist'.
This raises a surprising but interesting question: will many-core be the Trojan horse of formal methods for developing critical software ?

\section{Multicores and manycores pose different certification problems}

Multicores and manycores look the same from an hardware point of view : cores sharing memories.
Therefore both undergo the same problems w.r.t. critical applications:
\begin{itemize}
\item
variability in access times to shared memory, which induces variability in execution times.
\item
race conditions: \cite{padua2011encyclopedia}
\begin{quotation}
when two or more threads access a common resource,
e.g., a variable in shared memory, and the order of the
accesses depends on the timing, i.e., the progress of
individual threads
\end{quotation}

\end{itemize}

For multicores this is significant, even with long pipelines and out-of-order execution. But with only slight exaggeration, this is their only problem with regard to certification.

For manycores, this variability is probably not as important, because the architecture is completely 'data-oriented'.
But let's say it is: it's still not the most feared phenomenon.
The problem comes from the particular kind of supported algorithm, which induces a particular kind of sharing between threads :
\begin{itemize}
\item on a multicore, in an embedded use, the design aims to safely assess this variability, for example with a synchronous approach.
In this case, each task has exclusivity on its own data, and data exchanged between tasks are carefully read and written at the start and end of the tasks, in such a way to avoid simultaneous accesses.
In other words the few data sharing which occurs is under time-control, and there is mainly a non-functional bus/memory sharing.
In brief, tasks share mainly organs, and there is absolutely no difficulty in obtaining a deterministic functional behavior.
\item on a manycore, hundreds of threads read and write simultaneously the same data arrays. Threads share not only organs, but also data. This leads to the famous \emph{data races}, where a thread reads a data which is updated simultaneously by another thread. This leads to non-determinism, which is usually a show-stopper for certification.
\end{itemize}

We will focus on OpenCL \cite{munshi2011opencl}.


\section{The OpenCL subset}


synchronization : host and device (wi, kernel)
work-item synchronization is not necessary in case of embarassingly parallel
applies only to work-items in the same work-group

\section{Issues and discussion}

bla

bla

skeletons : \cite{cole2004bringing}, \cite{steuwer2011skelc}

\cite{betts2012gpuverify}

Early GPUs were primarily tailored toward \emph{embarrassingly parallel} graphics workloads :
computing independently each of the pixels that constitute the display, hence a low degree of data sharing.

\section{Related works}

bla

Two common ways to do so are sequential consistency
\cite{lamport1979make} and linearizability \cite{herlihy1990linearizability}. Both require that
the values returned by the operations appear to have
been returned by a sequential execution of the same
operations; sequential consistency only requires this
order to be consistent with the order in which each
individual thread invokes the operations, while linearizability
further requires this order to be consistent
with the real-time order of nonoverlapping operations.

bla

\section{Conclusions}

bla

bla

bla

\bibliography{opencl_certif}

\end{document}